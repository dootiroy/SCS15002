% This is a TEX command file that can be used as a skeleton
% for preparing reports for CAS consulting projects.
% A percent sign indicates the beginning of a comment,
% so TEX will ignore the remainder of the line.  When
% you need a real percent sign, type  \%.
%
\documentclass[12pt]{article}
\usepackage{epsfig,latexsym,amsfonts,amssymb,amsmath,multirow,lscape}
\hoffset=-0.875in
\advance\topmargin by -0.75truein
\oddsidemargin=0.875truein
\evensidemargin=0.875truein
\advance\textheight by 1.25truein
\textwidth=6.5truein
\parskip 10pt  % This puts some empty space between paragraphs.
\parindent 0pt   % This says paragraphs are not to be indented.
\begin{document}
%\pagestyle{empty}

\begin{flushleft}{\bf Statistical Consulting}\hfill
                 {\bf Spring 2014}  %{\bf Fall 2010}
\end{flushleft}

\begin{center}
 {\bf Project Report}
\end{center}

\vspace{.5cm}

\parbox{3.5in}{{\bf Client:} Santry, Laura  }\ \
              {{\bf File Number:}\ \ SCS14010 }

\vspace{.2cm}
\parbox{3.5in}{{\bf Department:} \ \ Physiology and Neurobiology, \\
                                  \ \ University of Connecticut}

\vspace{.2cm}
\parbox{3.5in}{{\bf Consultant:} Jing Wu\ \ }
              {{\bf Date of Initial Meeting:}\ \ 03/12/2014}

\vspace{.2cm} {\bf Persons Attending: }     \\  Santry, Laura   \ \  \\ Dr. Chen \ \ \\   Jing Wu\ \


\vspace{.2cm} {\bf Title of Project:} \\
A study on the relationship between the position of lumbar puncture and success rate.

\vspace{.5cm}

{\bf Statement of Problem:}
The client wants to determine if the position of the patient during a lumbar puncture has any relationship with the success rate of the procedure. She also wants to know the relationship between position of patient and quality of cerebral spinal fluid.

\vspace{.5cm}

{\bf Background:}
The client is writing her Honors Senior Thesis based upon a subcomponent of a large, multi-institutional clinical study regarding the evaluation of the “Just In Time” JIT Simulator Performance Training Program on the clinical success rates of lumbar punctures in infants. Her research question seeks to investigate if the position of the patient during a lumbar puncture has any correlation with the success rate of the procedure.

\vspace{.5cm}
{\bf Major Activities and Progress:}
In the first meeting, the client described the variables to be measured, i.e. position of patient, number of attempts necessary for a successful tap and quality of cerebral spinal fluid (numerical scale0-5). Success rate is based on the number of attempts for a successful lumbar puncture and quality (descriptive) of the fluid obtained. The client wants to make a design for this project. Chi-squared test for categorical data is suggested. The client is still waiting for the data.

 \vspace{0.5cm}
{\bf Status:}
Incomplete.



\end{document}
